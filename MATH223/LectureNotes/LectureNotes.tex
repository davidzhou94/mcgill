\documentclass{report} 
\usepackage{amsmath}
\usepackage{amsthm,amssymb,graphicx}
\graphicspath{ {F:\repos\mcgill\MATH223\LectureNotes} }

\newtheorem{_thm}{Theorem}
\theoremstyle{definition}
\newtheorem{_prob}{Problem}
\newtheorem{_sol}{Solution}
\newtheorem{_def}{Definition}
\newtheorem{_rem}{Remark}

\begin{document} 
\title{MATH 223, Winter 2015} 
\author{Yang David Zhou}
\maketitle

\chapter{Introduction}

\section{Administrativa}

\raggedright

Professor Tiago Salvador \newline

Office: BH1036

Office Hours: W4:15-5:45PM, F3:00-4:30PM \newline

\textbf{Grading}

\begin{tabular}{ l l l }
  Assignments & 15\% & 15\% \\
  Midterm     & 25\% &  0\% \\
  Final       & 60\% & 85\% \\
\end{tabular} \newline

The midterm will be scheduled for the 7th week of class.

\section{A Review of Vectors}

\subsection{Vectors in \(\mathbb{R}^n\)}

\(\mathbb{R}^n\) is the set of all \(n\)-tuples of real numbers \(u=(a_1 ... a_n) \mid a\in \mathbb{R}\) where \(a\) are the \textbf{components} or \textbf{entries}.

\begin{_rem}
We use the term \textbf{scalar} to refer to an element in \(\mathbb{R}\).
\end{_rem}

\subsection{Basic Definitions}

\textbf{Addition}

\(u, v \in \mathbb{R}^n\)

\(u=(a_1...a_n)\) 

\(v=(b_1...b_n)\)

\(u+v=(a_1+b_1...a_n+b_n)\) \newline

\textbf{Scalar Multiplication}

\(k\in \mathbb{R}\)

\(ku=(ka_1...ka_n)\) \newline

\textbf{Equality}

Two vectors \(u\) and \(v\) are said to be equal (\(u=v\)) if \(a_i=b_i \forall i=1...n\). \newline

\textbf{Zero Vector}

The zero vector is defined as \(0=(0...0)\). \newline

\textbf{Linear Combination}

Suppose we are given \(m\) vectors \(u_1...u_m\in \mathbb{R}^n\) and \(m\) scalars \(k_1...k_m\in \mathbb{R}\).

Let \(u=k_1u_1+...+k_mu_m\).

Such a vector \(u\) is called a linear combination of the vectors \(u_1...u_m\). \newline

\textbf{Vector Multiple}

A vector \(u\) can be called a multiple of \(v\) if there is a scalar \(k\) such that \(u=kv\) with \(k\neq 0\). 
In the case \(k>0\) we say \(u\) is in the same direction as \(v\). 
In the case \(k<0\) we say \(u\) is in the opposite direction of \(v\). \newline

\subsection{The Dot Product}

\begin{_def}
Let \(u=(a_1...a_n)\) and \(v=(b_1...b_n)\). The \textbf{dot product} or inner product is given by,
\[u\cdot v=a_1b_1+...a_nb_n=\]
\end{_def}

\begin{_def}
The vectors \(u\) and \(v\) are \textbf{orthogonal} if \(u\cdot v=0\).
\end{_def}

\subsection{The Vector Norm}

\begin{_def}
The \textbf{norm} or \textbf{length} of a vector is given by,
\[\|u\|=\sqrt{a^2_1+...+a^2_n}\]
\end{_def}

Thus \(\|u\|\geq 0\) and \(\|u\|=0\) if and only if (iff) \(u=0\).

\begin{_def}
A vector is called a \textbf{unit vector} if \(\|u\|=1\).
\end{_def}

For any non-zero vector \(v\), the vector 
\[\hat{v}=\frac{1}{\|v\|} v\]
is the only unit vector with the same direction of $v$.
The process of finding \(\hat{v}\) is called \textbf{normalizing}.

\subsection{Theorem: Cauchy-Schwarz Inequality}

\begin{_thm}
Given any two vectors \(u,v\in \mathbb{R}^n\), then,
\[|u\cdot v|\leq \|u\|\|v\|\]
\end{_thm}

\begin{proof}
Let \(t\in \mathbb{R}\). So, \(\|tu+v\|^2\geq 0\).
\begin{align*}
\|tu+v\|^2 &= (tu+v)(tu+v) \\
&= (tu\cdot tu)+(tu\cdot v)+(v\cdot tu)+(v\cdot v) \\
&= t^2(u\cdot u)+t(v\cdot u)+t(u\cdot v)+(v\cdot v) \\
&= t^2\|u\|^2+2t(u\cdot v)+\|v\|^2
\end{align*}
We can represent this in the form \(at^2+bt+c\geq 0\), so,
\[a=\|u\|^2, b=2(u\cdot v), c=\|v\|^2\]
Take the Discriminant as \(b^2-4ac\iff b^2\leq 4ac\).
\begin{align*}
4(u\cdot v)^2 &\leq 4\|u\|^2\|v\|^2 \\
|u\cdot v| &\leq \|u\|\|v\|
\end{align*}
\end{proof}

\subsection{Theorem: Minkowski Triangle Inequality}

\begin{_thm}
Given \(u,v\in \mathbb{R}^n\), then \(\|u+v\|\leq \|u\|+\|v\|\).
\end{_thm}

\begin{proof}
\begin{align*}
\|u+v\|^2 &= \|u\|^2+2(u\cdot v)+\|v\|^2 \\
&\leq \|u\|^2+2\|u\|\|v\|+\|v\|^2 \text{ by C-S inequality} \\
&= (\|u\|+\|v\|)^2
\end{align*}
So, \(\|u+v\|^2\leq (\|u\|+\|v\|)^2\).
Take the square root and we are done.
\end{proof}

\subsection{Geometry with Vectors}

\begin{_def}
The \textbf{distance} between vectors \(u,v\in \mathbb{R}^n\) is given by,
\[d(u,v)=\|u-v\|=\sqrt{(a_1-b_1)^2+...+(a_n-b_n)^2}\]
\end{_def}

\begin{_def}
The \textbf{angle} between vectors \(u,v\in \mathbb{R}^n\) is given by,
\[cos\theta =\frac{u\cdot v}{\|u\|\|v\|} \quad \theta \in [0,\pi]\]
\end{_def}

Observe that in the previous definition, the angle is well defined.

\[-\|u\|\|v\|\leq -|u\cdot v|\leq u\cdot v\leq u\cdot v\leq |u\cdot v|\leq \|u\|\|v\|\]

Dividing the entire inequality by $\|u\|\|v\|$ yields,

\[-1\leq \frac{u\cdot v}{\|u\|\|v\|} \leq 1\]

\begin{_def}
A \textbf{hyperplane} $\mathcal{H}$ in $\mathbb{R}^n$ is the set of points $(x_1...x_n)$ that satisfy $a_1x_1+...+a_nx_n=b$ where $u=[a_1...a_n]\in \mathbb{R}^n$ and $b\in \mathbb{R}$.
\end{_def}

\begin{_def}
The \textbf{line} in $\mathbb{R}^n$ passing through a point $P=(b_1...b_n)$ and in the direction of $v\in \mathbb{R}^n$ with $v\neq 0$.
\[x=P+tu \quad t\in \mathbb{R}, \quad u=[a_1...a_n]\]
\[ \left\{ 
  \begin{array}{l}
    x_1=a_1t+b_1 \\
    x_n=a_nt+b_n
  \end{array} \right.\]
\end{_def}

\section{Algebra of Matrices}

\end{document}