\documentclass{report} 

\usepackage{amsmath,amsthm,amssymb,graphicx}

\graphicspath{ {C:/Users/David/SkyDrive/School/MATH 223/Lecture Notes/} }

\newtheorem{myprob}{Problem} 

\theoremstyle{definition} 

\newtheorem*{solution}{Solution}

\newtheorem{mydef}{Definition}

\newtheorem{myrem}{Remark}

\begin{document} 

\title{MATH 223, Winter 2015} 

\author{Yang David Zhou}

\maketitle

\chapter{Introduction}

\section{Administrativa}

\raggedright

Professor Tiago Salvador \newline

Office: BH1036

Office Hours: W4:15-5:45PM, F3:00-4:30PM \newline

\textbf{Grading}

\begin{tabular}{ l l l }
  Assignments & 15\% & 15\% \\
  Midterm     & 25\% &  0\% \\
  Final       & 60\% & 85\% \\
\end{tabular} \newline

The midterm will be scheduled for the 7th week of class.

\section{Review}

\subsection{Vectors in \(\mathbb{R}^n\)}

\(\mathbb{R}^n\) is the set of all \(n\)-tuples of real numbers \(u=(a_1 ... a_n) \mid a\in \mathbb{R}\) where \(a\) are the \textbf{components} or \textbf{entries}.

\begin{myrem}
We use the term \textbf{scalar} to refer to an element in \(\mathbb{R}\).
\end{myrem}

\subsection{Basic definitions}

\textbf{Addition}

\(u, v \in \mathbb{R}^n\)

\(u=(a_1...a_n)\) 

\(v=(b_1...b_n)\)

\(u+v=(a_1+b_1...a_n+b_n)\) \newline

\textbf{Scalar Multiplication}

\(k\in \mathbb{R}\)

\(ku=(ka_1...ka_n)\) \newline

\textbf{Equality}

Two vectors \(u\) and \(v\) are said to be equal (\(u=v\)) if \(a_i=b_i \forall i=1...n\). \newline

\textbf{Zero Vector}

The zero vector is defined as \(0=(0..0)\). \newline

\textbf{Linear Combination}

Suppose we are given \(m\) vectors \(u_1...u_m\in \mathbb{R}^n\) and \(m\) scalars \(k_1...k_m\in \mathbb{R}\).

Let \(u=k_1u_1+...+k_mu_m\).

Such a vector \(u\) is called a linear combination of the vectors \(u_1...u_m\). \newline

\textbf{Vector Multiple}

A vector \(u\) can be called a multiple of \(v\) if there is a scalar \(k\) such that \(u=kv\) with \(k\neq 0\). In the case \(k>0\) we say \(u\) is in the same direction as \(v\). In the case \(k<0\) we say \(u\) is in the opposite direction of \(v\). \newline

\subsection{The Dot Product}

\begin{mydef}
Let \(u=(a_1...a_n\) and \(v=b_1...b_n\). The \textbf{dot product} or inner product is given by:

\(u\cdot v=a_1b_1+...a_nb_n=

\end{document}