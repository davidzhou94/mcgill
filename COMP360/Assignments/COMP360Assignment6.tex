\documentclass{article} 

\usepackage{amsmath,amsthm,amssymb,graphicx}

\graphicspath{ {C:/Users/David/SkyDrive/School/COMP 360/Assignments/} }

\newtheorem{problem}{Problem} 

\theoremstyle{definition} 

\newtheorem*{solution}{Solution} 

\begin{document} \title{Assignment 6} 

\author{Yang David Zhou, ID 260517397} 

\date{\today}

\maketitle

\begin{problem} 

Hitting Set.

\end{problem}

\begin{solution}

Use this algorithm:

For all \(m\) \(B_i=a_1,...,a_c\), remove each one of the \(a_i\) from \(A
\) and recursively call the algorithm with \(k=k-1\). If at least one of the \(a_i\) in each of the \(B_i\) returns a \(k-1\) size hitting set then there is a \(k\) size hitting set.

The running time is \(T(m,k)=c\cdot (T(m,k-1)+m)\) which is bounded by \(c^k\cdot c\cdot k\cdot m\cdot\). So \(f(c,k)=c^k\cdot c\cdot k\) and \(p(n,m)=m\).

This algorithm works because, \(\exists H, |H|=k \) if and only if \(\exists a_i\in B_i \forall B_i\subseteq A\) such that removing \(a_i\) from this instance of the problem can give a new hitting set \(H', |H'|=k-1\). This is true because 

\end{solution}

\begin{problem} 

3-SAT.

\end{problem}

\begin{solution}

Given the constraint that each variable in this class of 3-SAT problems must appear once in exactly 3 clauses and every clause contains exactly 3 variables, there are an equal number of clauses are variables.

We can represent this problem as a graph \(G\). Let \(X=x_1...x_n\) be the set of variables and \(K=k_1...k_m\) be the set of clauses. Create a node \(v_i\) for every \(x_i\in X\) and a node \(c_j\) for every \(k_j\in K\) and an edge \((v_i,c_j)\) if \(x_i\) appears in \(k_j\). \(G\) is bipartite as there are no edges between within the \(v_i\) or \(c_j\). \(G\) contains a perfect matching. This is true because if \(G\) did not contain a perfect matching then there must be at least one \(x_i\) that does not appear in exactly 3 clauses which is a contradiction. Since \(G\) contains a perfect matching, then every variable can be set to whatever satisfies it's perfect matching clause. Since the matching is perfect, all clauses are satisfied. This satisfying assignment can be found in poly-time as finding a perfect matching is poly-time in input size and building the graph is poly-time in input size.

\end{solution}

\begin{problem} 

Claws.

\end{problem}

\begin{solution}

We can represent this problem as a kind of 2-colouring problem. Start with any 2 colour assignment of the nodes. Check if the 2 colour assignment contains \(k\) claws, so the central vertex \(x\) is colour 1 and there are 3 adjacent vertices \(y_1,y_2,y_3\) of colour 2. If there are \(k\) claws then we have found the maximum packing. If there are not, then we try a new colour and repeat.

This randomised algorithm has a high probability of success because there are at most \(n \choose 2\)\( = \frac{1}{2} n\cdot (n-1) \) colour assignments. Any random assignment is correct with probability \(P(trials)=trials\cdot \frac{2}{n(n-1)}\). We are guaranteed to find the correct assignment after \(\frac{1}{2} n^2 - n\) trials.

This algorithm will succeed after a polynomial number of runs and likely before all the possible colourings are tested.

\end{solution}

\begin{problem} 

Monotone QSAT.

\end{problem}

\begin{solution}

\end{solution}

\begin{problem} 

Geography.

\end{problem}

\begin{solution}

\end{solution}

\begin{problem} 

Randomised Minimum s-t Cut.

\end{problem}

\begin{solution}

\end{solution}

\end{document}