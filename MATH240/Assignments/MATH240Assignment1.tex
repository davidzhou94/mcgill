\documentclass{article} 

\usepackage{amsmath,amsthm,amssymb,tikz}

\usetikzlibrary{shapes,backgrounds}

\newtheorem{problem}{Problem} 

\theoremstyle{definition} 

\newtheorem*{solution}{Solution} 

\begin{document} \title{Assignment 1} 

\author{Yang David Zhou, ID 260517397} 

\date{\today}

\maketitle

\begin{problem} 

Venn Diagrams. Draw the Venn diagrams for 

(a) \( (A \oplus B)\cap C \)

(b) \( \bar{A} \cap \bar{B} \cap \bar{C} \)

\end{problem}

\begin{solution}

\def\firstcircle{(90:1cm) circle (1.5cm)}
\def\secondcircle{(210:1cm) circle (1.5cm)}
\def\thirdcircle{(330:1cm) circle (1.5cm)}

(a)

\begin{tikzpicture}

\filldraw [color=black, fill=white, thick] (-3,-3)rectangle (3,3);

\begin{scope}
\clip \secondcircle;
\fill[cyan] \thirdcircle;
\end{scope}

\begin{scope}
\clip \firstcircle;
\fill[cyan] \thirdcircle;
\end{scope}

\begin{scope}
\clip \firstcircle;
\clip \secondcircle;
\fill[white] \thirdcircle;
\end{scope}

\draw \firstcircle node [text=black,above] {$A$};
\draw \secondcircle node [text=black,below left] {$B$};
\draw \thirdcircle node [text=black,below right] {$C$};

\end{tikzpicture}

(b)

\begin{tikzpicture}

\filldraw [color=black, fill=cyan, thick] (-3,-3)rectangle (3,3);

\begin{scope}
\fill[white] \firstcircle;
\fill[white] \secondcircle;
\fill[white] \thirdcircle;
\end{scope}

\draw \firstcircle node [text=black,above] {$A$};
\draw \secondcircle node [text=black,below left] {$B$};
\draw \thirdcircle node [text=black,below right] {$C$};

\end{tikzpicture}

\end{solution}

\begin{problem}

Set Identities. Prove the following

(a) \( \neg (A \cap B) = \neg A \cup \neg B \)

(b) \( (B-A)\cup (C-A)=(B\cup C)-A \)

\end{problem}

\begin{solution}

(a)

\begin{proof}

I will show that \( \neg (A \cap B) \subset \neg A \cup \neg B \) in i) 
then \( \neg A \cup \neg B \subset \neg (A \cap B) \) in ii) \\ \\

i) \\
Take any \(x \in \neg (A \cap B) \) \\
So, \(x \in U - A \cap B \) \\
So, \(x \in U \) and \(x \notin A \cap B \) \\
If \(x \notin A \cap B \), then \(x \in U - A \) and \(x \in U - B \) \\
Therefore, \(x \in \neg A \cup \neg B \) \\ \\

ii) \\
Take any \(y \in \neg A \cup \neg B \) \\
Either \(y \in \neg A \subset \neg (A \cap B) \) or \(y \in \neg B \subset \neg (A \cap B) \) \\
Therefore, \(y \in \neg (A \cap B) \)

\end{proof}

(b)

\begin{proof}

I will show that \( (B-A)\cup (C-A) \subset (B\cup C)-A \) in i) 
then \( (B\cup C)-A \subset (B-A)\cup (C-A) \) in ii) \\ \\

i) \\
Take any \(x \in (B-A)\cup (C-A) \) \\
So, either \(x \in (B-A) \) or \( x \in (C-A) \)\\
Since, \(x \in (B-A) \subset (B \cup C)-A \) because \(B \subset B \cup C \)

 and \(x \in (C-A) \subset (B \cup C)-A \) because \(C \subset B \cup C \)
 
 and both \(B-A \) and \(C-A \) are without \(A\) \\
Therefore, \(x \in (B \cup C)-A \) \\ \\

ii) \\
Take any \(y \in (B \cup C)-A \) \\
So, \(y \in B \cup C \) but \(y \notin A \) \\
If either \(y \in B \) or \(y \in C \) and \(y \notin A \) then \(y \in (B-A) \cup (C-A) \)

\end{proof}

\end{solution}

\begin{problem}

Propositions. Which of the following sentences are statements?

(a) Montréal is an island 

(b) \(6+5=10\)

(c) \(x+5=10\)

\end{problem}

\begin{solution}

All of the sentences are statements except (c) because (a) and (b) are true and false respectively whereas c depends upon \(x\) and cannot be evaluated to be true or false. A statement or proposition must be either determinately true or false.

\end{solution}

\begin{problem}

Conditional Statements. Which of the following implications are true?

(a) If \(1 + 1 = 2\) then pigs can fly.

(b) If pigs can fly then \(1 + 1 = 2\).

(c) If \(1 + 1 = 3\) then pigs can fly.

(d) If pigs can fly then \(1 + 1 = 3\).

\end{problem}

\begin{solution}

Given an implication of the form "if \(p\) then \(q\)" or \(p=>q\), we can assess the validity of each implication by assuming that the statements "pigs can fly" and \(1+1=3\) are false while \(1+1=2\) is true.

The implications by letter are: 

(a) \(p=true\) \(q=false\) therefore, FALSE.

(b) \(p=false\) \(q=true\) therefore, TRUE.

(c) \(p=false\) \(q=false\) therefore, TRUE.

(d) \(p=false\) \(q=false\) therefore, TRUE.

\end{solution}

\begin{problem}

Tautologies. Which of the following are tautologies? If the statement is a
tautology give a proof using the appropriate rules of logic at each step of the proof. If not, then justify your answer by giving a counter-example or using a proof table.

(a) \(p \Rightarrow (p \vee q) \)

(b) \( \neg (p \Rightarrow q) \equiv \neg q \)

(c) \( \neg (p \oplus q) \equiv (p \Leftrightarrow q) \)

(d) \( ((p\Rightarrow q)\Rightarrow r) \equiv (p\Rightarrow (q\Rightarrow r)) \)

(e) \( (\neg p \wedge (p\Rightarrow q)) \equiv \neg q \)

\end{problem}

\begin{solution}

(a) IS a tautology:

\begin{proof}

\(p \Rightarrow (p \vee q) \)

\(\equiv \neg p \vee (p \vee q) \)

\(\equiv (\neg p \vee p) \vee q \)

\(\equiv 1 \vee q \)

\(\equiv 1 \)

\end{proof}

(b) IS NOT a tautology:

When \(p\) and \(q\) are both false, \(\neg (p\Rightarrow q)\) evaluates to false while \(\neg q\) evaluates to true. \\

(c) IS a tautology:

\begin{proof} I will show that \(p \oplus q\) is equivalent to \(p \Leftrightarrow q\):

\(p \oplus q\)

\(\equiv \neg ((p \vee q)\wedge \neg (p \wedge q)) \)

\(\equiv \neg (p \vee q)\vee (p \wedge q) \)

\(\equiv p \Leftrightarrow q \)

\end{proof} 

(d) IS NOT a tautology:

When \(p\), \(q\) and \(r\) are all false, \( (p \Rightarrow q) \Rightarrow r \) evaluates to false while \\ \( p \Rightarrow (q \Rightarrow r ) \) evaluates to true. \\

(e) IS NOT a tautology:

When \(p\) is false and \(q\) is true, \(\neg p \wedge (p\Rightarrow q)\) evaluates to true while \(\neg q\) evaluates to false.

\end{solution}

\begin{problem}

Circuits. Show how NAND gates can be used to simulate OR, AND and NOT gates.

\end{problem}

\begin{solution}

I will represent a NAND gate with this pattern: \( \neg ( p \wedge q ) \) where the inputs are represented as statements \(p\) and \(q\). \\

OR Gate

\(a \vee b \equiv \neg (\neg (a \wedge a) \wedge \neg (b \wedge b) ) \) \\

AND Gate

\(a \wedge b \equiv \neg (\neg (a \wedge b) \wedge \neg (a \wedge b) ) \) \\

NOT Gate

\(\neg a \equiv \neg (a \wedge a) \)

\end{solution}

\end{document}