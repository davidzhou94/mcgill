\documentclass{article} 

\usepackage{amsmath,amsthm,amssymb}

\newtheorem{problem}{Problem} 

\theoremstyle{definition} 

\newtheorem*{solution}{Solution} 

\begin{document} \title{Assignment 4} 

\author{Yang David Zhou, ID 260517397} 

\date{\today}

\maketitle

\begin{problem} 

Primality Testing.

\end{problem}

\begin{solution}

(a) \(a=5\), \(n=124\) in \(a^{n-1} \equiv 1\mod n\)

\(5^{124-1} \mod 124\)

Since \(5\) is clearly not divisible by \(124\), we can apply Fermat's Little Theorem.

\(\equiv 1 \mod 124\)

For \(a=5\), \(n=124\) passes the test. \\

(b) The test did not give the correct answer since \(124\) is clearly an even number and therefore not prime. This is an example of a liar number and the reason why the Fermat Primality Test is probabilistic.

\end{solution}

\begin{problem} 

RSA Encryption.

\end{problem}

\begin{solution}

(a) Encryption is \(\hat{M}=M^p \mod n\) with \(\{n=91,p=5\}\)

\(\hat{M}=4^5 \mod 91\)

\(=4^4\cdot 4 \mod 91\)

\(=(4^4 \mod 91 \cdot 4 \mod 91) \mod 91\)

\(=((4^2 \mod 91 \cdot 4^2 \mod 91) \mod 91 \cdot 4 \mod 91) \mod 91\)

\(=(74 \cdot 4) \mod 91\)

\(=296 \mod 91\)

\(\hat{M} =23\) \\

(b) We use the modular inverse of \(p \mod (q_1-1)(q_2-2)\) so \(x=5^{-1}\mod 72\) \\

\(\gcd (72,5)\)

\(=\gcd (5,2)\)

\(=\gcd (2,1)\)

\(=\gcd (1,0)\)

\(=1\) \\

\(1=5-2\cdot 2\)

\(=5-2\cdot (72-14\cdot 5)\)

\(1=29\cdot 5-2\cdot 72)\)

So we use \(x=29\) \\

(c) Decryption is \(M=\hat{M}^x \mod n\) with \(n=91,x=29\)

\(M=23^{29} \mod 91\)

\(M=23^{16}\cdot 23^{8}\cdot 23^{4}\cdot 23 \mod 91\) \\

[Aside]

\(23^{2} \mod 91=74\)

\(23^{4} \mod 91=(23^{2} \mod 91 \cdot 23^{2} \mod 91) \mod 91=16\)

\(23^{8} \mod 91=(23^{4} \mod 91 \cdot 23^{4} \mod 91) \mod 91=74\)

\(23^{16} \mod 91=(23^{8} \mod 91 \cdot 23^{8} \mod 91) \mod 91=16\) \\

\(M=16\cdot 74\cdot 16\cdot 23 \mod 91\)

\(M=16\cdot 23 \mod 91\)

\(M=368 \mod 91\)

\(M=4\)

\end{solution}

\begin{problem} 

A Combinatorial Identity.

\end{problem}

\begin{solution}

(a) We observe that,

\(\binom{n}{0}\cdot 2^0+\binom{n}{1}\cdot 2^1+...+\binom{n}{n}\cdot 2^n=3^n\)

Can be rewritten as,

\((1+2)^n=\sum\limits_{k=0}^n \binom{n}{k} 2^k\)

Which is simply the binomial theorem when \(x=1,y=2\). The equality in the binomial theorem was proven in class.

(b) In the form,

\(3^n=\sum\limits_{k=0}^n \binom{n}{k} 2^k\)

The LHS can be viewed as counting the number of possible sequences in a \(n\)-tumbler combination lock where each tumbler is either \(\{0,1,2\}\).

The RHS also counts the possible sequences. The number of ways to choose \(k\) tumblers that is either a \(0\) or a \(1\) is \(\binom{n}{k}\). In each of these choices there are a further \(2^k\) ways to assign a \(0\) or a \(1\) to the tumblers. So the term \(\binom{n}{k} 2^k\) counts both the combinations of having \(k\) \(0\)s and \(1\)s and the \(n-k\) tumblers that have a \(2\) since \(2\) is the only choice possible for the unassigned \(n-k\) tumblers. Summing up the \(k=0...n\) terms gives all the possible sequences.

\end{solution}

\end{document}