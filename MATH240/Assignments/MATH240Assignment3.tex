\documentclass{article} 

\usepackage{amsmath,amsthm,amssymb}

\newtheorem{problem}{Problem} 

\theoremstyle{definition} 

\newtheorem*{solution}{Solution} 

\begin{document} \title{Assignment 3} 

\author{Yang David Zhou, ID 260517397} 

\date{\today}

\maketitle

\begin{problem} 

Prime Factorisation.

\end{problem}

\begin{solution}

(a) Prime Factorisation of \(419\)

\(419\) is a prime number \\

(b) Prime Factorisation of \(9555\)

\(9555=3\cdot 5\cdot 7^2\cdot 13\) \\

(c) Prime Factorisation of \(10!\)

\(10!=1\cdot 2\cdot 3\cdot 4\cdot 5\cdot 6\cdot 7\cdot 8\cdot 9\cdot 10\)

\(10!=2^8\cdot 3^4\cdot 5^2\cdot 7\)

\end{solution}

\begin{problem}

Euclid's Algorithm.

\end{problem}

\begin{solution}

(a) Find \(d=\gcd (177,38)\) \\

\(\gcd (177,38)=\gcd (38,25)\)

\(\gcd (38,25)=\gcd (25,13)\)

\(\gcd (13,12)=\gcd (12,1)\)

\(\gcd (12,1)=\gcd (1,0)\) 

\(d=1\) \\

(b) Find \(s,t\in \mathbb{Z}\) such that \(d=38s+177t\) \\

\(d=1\)

\(d=13-12\)

\(d=13-(25-13)=2\cdot 13-25\)

\(d=2\cdot (38-25)-25=2\cdot 38-3\cdot 25\)

\(d=2\cdot 38-3\cdot (177-4\cdot 38)=14\cdot 38-3\cdot 177\)

In \(d=38(14)+177(-3)\), we have \(s=14,t=-3\)

\end{solution}

\begin{problem}

Greatest Common Divisors.

\end{problem}

\begin{solution}

(a) Suppose that \(\gcd (a,y)=1\) and \(\gcd (b,y)=d\). Prove that \(\gcd (a\cdot b,y)=d\). \\

By Bezouts' Lemma we have the following:

(1) \(\gcd (b,y)=d=sb+ty\)

(2) \(\gcd (a,y)=1=s'a+t'y\)

Where \(s,t,s',t'\in \mathbb{Z}\)

So we can take (1) and multiply all the terms in it by \(1\),

\(d(1)=sb(1)+ty(1)\)

And substitute with (2),

\(d=sb(s'a+t'y)+ty\)

And rearrange,

\(d=sbs'a+sbt'y+ty\)

\(d=ss'ab+(sbt'+t)y\) \\

In the definition, \(s,s',t,t',b\) are all integers. Thus, we can show that \(d\) as the sum of \(ab\) and \(y\) each multiplied by an integer, i.e.,

\(\gcd (ab,y)=i\cdot (a\cdot b)+j\cdot y\) \\

Finally we know that \(d\) is identical in (1) and in \(d=ss'ab+(sbt'+t)y\)\\

(b) Suppose that \(\gcd (b,a)=1\). Prove that \(\gcd (b+a,b-a)\leq 2\).



\end{solution}

\begin{problem}

Pseudorandom Numbers.

\end{problem}

\begin{solution}

(a) \(x_{k+1}=11x_k+37\mod 100\) with seed \(x_0=52\)

\(x_0=52\)

\(x_1=11(52)+37\mod 100=9\)

\(x_2=11(9)+37\mod 100=36\)

\(x_3=11(36)+37\mod 100=33\)

\(x_4=11(33)+37\mod 100=0\)

\(x_5=11(0)+37\mod 100=37\)

\(x_6=11(37)+37\mod 100=44\)

\(x_7=11(44)+37\mod 100=21\)

\(x_8=11(21)+37\mod 100=68\)

\(x_9=11(68)+37\mod 100=85\)

\(x_{10}=11(85)+37\mod 100=72\) \\

(b) \(x_{k+1}=8x_k+24\mod 128\) with seed \(x_0=0\)

\(x_0=0\)

\(x_1=8(0)+24\mod 128=24\)

\(x_2=8(24)+24\mod 128=88\)

\(x_3=8(88)+24\mod 128=88\)

\(x_4=8(88)+24\mod 128=88\)

\(x_5=8(88)+24\mod 128=88\)

\(x_6=8(88)+24\mod 128=88\)

\(x_7=8(88)+24\mod 128=88\)

\(x_8=8(88)+24\mod 128=88\)

\(x_9=8(88)+24\mod 128=88\)

\(x_{10}=8(88)+24\mod 128=88\)

\end{solution}

\begin{problem}

Modular Equations.

\end{problem}

\begin{solution}

Solve for \(x\) in \(169x=10\mod 419\) with the modular inverse of \(169\). \\

\(\gcd (419,169)\)

\(=\gcd (169,81)\)

\(=\gcd (81,7)\)

\(=\gcd (7,4)\)

\(=\gcd (4,3)\)

\(=\gcd (3,1)\)

\(=\gcd (1,0)=1\) \\

\(1=1(3)-2(1)\)

\(1=1(3)-2(4-3)=3(3)-2(4)\)

\(1=3(7-4)-2(4)=3(7)-5(4)\)

\(1=3(7)-5(81-11(17))=58(7)-5(81)\)

\(1=58(169-2(81))-5(81)=58(169)-121(81)\)

\(1=58(169)-121(419-2(169))\)

\(1=300(169)-121(419)\)

\(169^{-1}=s=300\mod 419\) \\

Now that we have obtained the modular inverse, we can solve the equation:

\(169x=10\mod 419\)

\(169^{-1}\cdot 169x=169^{-1}\cdot 10\mod 419\)

\(x=300\cdot 10\mod 419\)

\(x=3000\mod 419\)

\(x=67\)

\end{solution}

\begin{problem}

Congruences.

\end{problem}

\begin{solution}

(a) Evaluate \(6022^{1267}\mod 17\)

Here we apply Fermat's Little Theorem to evaluate,

\(6022^{1267}\mod 17\)

\(6022\) can be rewritten as \(6022=354\cdot 17+4\), so by property of modulus,

\(=4^{1267}\mod 17\)

\(=2^{2534}\mod 17\)

\(=2^{158(16)+6}\mod 17\)

\(=(2^{16})^{158}\cdot 2^6\mod 17\)

\(=((2^{16}\mod 17)^{158}\cdot 2^6\mod 17)\mod 17\)

Since \(2\nmid 17\) as clearly \(\gcd (17,2)=1\),

\(=((1)^{158}\cdot 64\mod 17)\mod 17\)

\(=((13)\mod 17\)

\(=13\) \\

(b) Evaluate \(3^{42637}\mod 419\)

Again, we apply FLT to evaluate,

\(3^{42637}\mod 419\)

\(=3^{102(418)+1}\mod 419\)

\(=(3^{418})^{102}\cdot 3^1\mod 419\)

\(=((3^{418}\mod 419)^{102}\cdot 3^1\mod 419)\mod 419\)

Since \(3\nmid 419\) as clearly \(\gcd (419,3)=1\),

\(=((1)^{102}\cdot 3)\mod 419\)

\(=3\mod 419\)

\(=3\)

\end{solution}

\end{document}