\documentclass{article} 

\usepackage{amsmath,amsthm,amssymb}

\newtheorem{problem}{Problem} 

\theoremstyle{definition} 

\newtheorem*{solution}{Solution} 

\begin{document} \title{COMP 302 Assignment 2 Question 1} 

\author{Yang David Zhou, ID 260517397} 

\date{\today}

\maketitle

\begin{problem} 

Prove reflect (reflect t)= t.

\end{problem}

\begin{proof}

Proof by induction: 

\begin{verbatim}
Base case: 
reflect Empty = Empty
reflect (reflect Empty) = Empty

Induction hypothesis:
Assume reflect (reflect someNode) = someNode for left and right
in someNode = Node (x, left, right)

let someTree = Node (x, left, right)
reflect someTree = Node (x, reflect right, reflect left)

reflect (reflect someTree) 
 = Node (x, reflect (reflect left), reflect (reflect right))
 = Node (x, left, right)   //from induction hypothesis
 = someTree
\end{verbatim}

\end{proof}

\begin{problem}

Prove that for all m:'a tree, size m = size' m 0.

\end{problem}

\begin{proof}

Auxiliary proof by induction, then show that the statement is a special case: 

\begin{verbatim}
Begin by proving size' someTree a = size someTree + a

Base case: 
size' Empty a 
 = a = 0 + a
 = size Empty + a

Induction hypothesis:
Assume size' someNode a = size someNode + a for left and right
in someNode = Node(x, left, right)

let someTree = Node (x, left, right)

size' someTree a 
 = size' left (size' right (a + 1))  //by function definition
 = size left + size' right (1 + a)   //from induction hypothesis
 = size left + size right + 1 + a    //apply IH again
 = size Node (x', left, right) + a   //by function definition

The value of x' is irrelevant so we can set it to x to obtain:

 = size someTree + a

Now that I have proven size' someTree a = size someTree + a, 
I can show that there is the special case when a = 0

size m = size' m 0
\end{verbatim}

\end{proof}

\end{document}